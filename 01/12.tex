\subsection{%
  Лекция \texttt{23.11.17}.%
}

\begin{theorem}[Условия Коши-Римана (условия аналитичности функции)]
\label{thr:C-R-cond}
  \(f(z) \colon D \to \CC\) аналитическая в \(z_0 \in D\) тогда и только тогда,
  когда функции \(u(x, y) = \Re f(z)\) и \(v(x, y) = \Im f(z)\) дифференцируемы,
  имеют непрерывные производные в точках \(x_0 = \Re z_0\) и \(y_0 = \Im z_0\) и
  выполнены условия:

  \begin{equation*}
    \partder{u}{x} = \partder{v}{y}
    \qquad
    \partder{u}{y} = -\partder{v}{x}
  \end{equation*}
\end{theorem}

\begin{proof}
  \suff{}
  \begin{equation*}
    \begin{aligned}
      \exists f'(z_0) \in \CC
    \\
      \lim_{\Delta z \to 0} \frac{\Delta f}{\Delta z}
      = \lim_{\Delta z \to 0} \frac{f(z_0 + \Delta z) - f(z_0)}{\Delta z}
    \\
      = \lim_{\Delta z \to 0} \frac{
          \prh[\Big]{u(x_0 + \Delta x, y_0 + \Delta y)
            + i v(x_0 + \Delta x, y_0 + \Delta y)}
          -
          \prh[\Big]{u(x_0, y_0) + i v(x_0, y_0)}
        }{(x + i y) - (x_0 + i y_0)}
    \end{aligned}
  \end{equation*}

  Выберем направление для \(\Delta z \to 0\). Пусть \(\Delta z = \Delta x \to
  0\), тогда

  \begin{equation*}
    \lim_{\Delta z \to 0} \frac{\Delta f}{\Delta z}
    = \lim_{\Delta x \to 0} \frac{
        u(x_0 + \Delta x, y_0) - u(x_0, y_0)
        + i v(x_0 + \Delta x, y_0) - i v(x_0, y_0)
      }{\Delta x}
    = \lim_{\Delta x \to 0} \frac{\Delta_x u}{\Delta x}
      - i \lim_{\Delta x \to 0} \frac{\Delta_x v}{\Delta x} 
    = u_x (x, y) - i v_x (x, y)
  \end{equation*}

  Аналогично, если \(\Delta = i \Delta y \to 0\), то

  \begin{equation*}
    \lim_{\Delta z \to 0}  \frac{\Delta f}{\Delta z}
    = v_y (x, y) - i u_y (x, y)
  \end{equation*}

  Итого имеем

  \begin{equation*}
    f' (z_0)
    = \partder{u}{x} + i \partder{v}{x}
    = \partder{v}{y} - i \partder{u}{y}
    \implies
    \begin{cases}
      \displaystyle \partder{u}{x} = \partder{v}{y} \\
      \displaystyle \partder{v}{x} = \partder{u}{y}
    \end{cases}
  \end{equation*}

  Дифференцируемость функций \(u(x, y)\) и \(v(x, y)\) следует из существования
  пределов частных приращений, а непрерывность производных следует из
  аналитичности функции \(f(z)\).

  \ness{}
  \(u(x, y)\) и \(v(x, y)\) дифференцируемы в точке \((x_0, y_0)\).

  \begin{equation*}
    \begin{aligned}
      \abs{\Delta z} = \sqrt{(\Delta x)^2 + (\Delta y)^2}
    \\
      \Delta u = \partder{u}{x} \Delta x + \partder{u}{y} \Delta y
      + \under{\smallo(\Delta z)}{\epsilon_1}
      \qquad
      \Delta v = \partder{u}{x} \Delta x + \partder{u}{y} \Delta y
        + \under{\smallo(\Delta z)}{\epsilon_2}
    \end{aligned}
  \end{equation*}

  Таким образом

  \begin{equation*}
    \frac{\Delta f}{\Delta z}
    = \frac{\Delta u + i \Delta v}{\Delta x + i \Delta y}
    = \frac{
        u_x \Delta x + u_y \Delta y
        + i v_x \Delta x + i v_y \Delta y
        + \epsilon_1 + \epsilon_2
      }{\Delta x + i \Delta y}
    = \frac{u_x \Delta x + i v_y \Delta y}{\Delta x + i \Delta y}
      + \frac{u_y \Delta y + i v_x \Delta x}{\Delta x + i \Delta y}
      + \frac{\epsilon_1 + \epsilon_2}{\Delta x + i \Delta y}
  \end{equation*}

  Используем условия Коши-Римана: \(v_y = u_x\) и \(u_y = i^2 v_x\).

  \begin{equation*}
    \frac{\Delta f}{\Delta z}
    = \frac{u_x \Delta x + i u_x \Delta y}{\Delta x + i \Delta y}
      + i \cdot \frac{i v_x \Delta y + v_x \Delta x}{\Delta x + i \Delta y}
      + \frac{\epsilon_1 + \epsilon_2}{\Delta x + i \Delta y}
    = \frac{u_x (\Delta x + i \Delta y)}{\Delta x + i \Delta y}
      + i \cdot \frac{v_x (i \Delta y + \Delta x)}{\Delta x + i \Delta y}
      + \frac{\epsilon_1 + \epsilon_2}{\Delta x + i \Delta y}
    = u_x + i v_x + \frac{\epsilon_1 + \epsilon_2}{\Delta x + i \Delta y}
  \end{equation*}

  Тогда

  \begin{equation*}
    \exists \lim_{\Delta z \to 0} \frac{\Delta f}{\Delta z}
    = u_x + i v_x
  \end{equation*}

  Причем \(u_x\) и \(v_x\) непрерывны по условию, а полученный предел конечен.
\end{proof}

\begin{remark}
  В условиях теоремы \ref{thr:C-R-cond} справедливо

  \begin{equation*}
    f'(z_0) = \partder{u}{x} (x_0, y_0) + i \partder{v}{x} (x_0, y_0)
  \end{equation*}
\end{remark}

\subheader{Условия Коши-Римана в полярной системе координат}

Зададим полярную систему координат.

\begin{equation*}
  \begin{cases}
    x = r \cos \phi \\
    y = r \sin \phi
  \end{cases}
  \qquad
  \begin{cases}
    r = \sqrt{x^2 + y^2} \\
    \phi = \arctan \frac{y}{x} \text{ + надо учесть знак}
  \end{cases}
\end{equation*}

\begin{lemma}
  В полярных координатах условия Коши-Римана имеют вид

  \begin{equation*}
    \partder{u}{r} = \frac{1}{r} \partder{v}{\phi}
    \qquad
    \partder{u}{\phi} = -r \partder{v}{r}
  \end{equation*}
\end{lemma}

\begin{proof}
  \begin{equation*}
    \begin{aligned}
      \partder{v}{\phi}
      = \partder{v}{x} \partder{x}{\phi} + \partder{v}{y} \partder{y}{\phi}
      = \partder{v}{x} (-r \sin \phi) + \partder{v}{y} r \cos \phi
    \\
      \partder{u}{r}
      = \partder{u}{x} \partder{x}{r} + \partder{u}{y} \partder{y}{r}
      = \partder{u}{x} \cos \phi + \partder{u}{y} \sin \phi
    \end{aligned}
  \end{equation*}

  Во втором полученном равенстве воспользуемся условиями Коши-Римана, получим

  \begin{equation*}
    \partder{u}{r}
    = \partder{u}{x} \cos \phi + \partder{u}{y} \sin \phi
    = \partder{v}{y} \cos \phi - \partder{v}{x} \sin \phi
    = \frac{1}{r} \partder{v}{\phi}
  \end{equation*}

  Доказательство второго равенства аналогично.

  \begin{equation*}
    \begin{aligned}
      \partder{v}{r}
      = \partder{v}{x} \partder{x}{r} + \partder{v}{y} \partder{y}{r}
      = \partder{v}{x} \cos \phi + \partder{v}{y} \sin \phi
    \\
      \partder{u}{\phi}
      = \partder{u}{x} \partder{x}{\phi} + \partder{u}{y} \partder{y}{\phi}
      = \partder{u}{x} (-r \sin \phi) + \partder{u}{y} r \cos \phi
    \end{aligned}
  \end{equation*}

  Во втором полученном равенстве воспользуемся условиями Коши-Римана, получим

  \begin{equation*}
    \partder{u}{\phi}
    = \partder{v}{y} (-r \sin \phi) - \partder{v}{x} r \cos \phi
    = -r \prh{\partder{v}{y} \sin \phi + \partder{v}{x} \cos \phi}
    = -r \partder{v}{r}
  \end{equation*}
\end{proof}

\begin{lemma}
  \begin{equation*}
    f'(z) = \prh{\partder{u}{r} + i \partder{v}{r}} \cdot \frac{r}{z}
  \end{equation*}
\end{lemma}

\begin{proof}
  \begin{equation*} \label{eq:complex-derivate-polar-1} \tag{1}
    f'(z)
    = \partder{u}{x} + i \partder{v}{x}
    = \prh{
        \partder{u}{r} \cdot \partder{r}{x}
        + \partder{u}{\phi} \cdot \partder{\phi}{x}
      }
      + i \prh{
        \partder{v}{r} \cdot \partder{r}{x}
        + \partder{v}{\phi} \cdot \partder{\phi}{x}
      }
    = \prh{\partder{u}{r} + i \partder{v}{r}} \cdot \partder{r}{x}
      + \prh{\partder{u}{\phi} + i \partder{v}{\phi}} \cdot \partder{\phi}{x}
  \end{equation*}

  Теперь заметим, что

  \begin{equation*} \label{eq:complex-derivate-polar-2} \tag{2}
    \begin{aligned}
      \partder{r}{x}
      = \partder{\sqrt{x^2 + y^2}}{x}
      = \frac{x}{\sqrt{x^2 + y^2}}
      = \frac{x}{r}
    \\
      \partder{\phi}{x}
      = \partder{\arctan{\frac{y}{x}}}{x}
      = \frac{1}{1 + \frac{y^2}{x^2}} \cdot \prh{-\frac{y}{x^2}}
      = -\frac{y}{x^2 + y^2}
      = -\frac{y}{r^2}
    \end{aligned}
  \end{equation*}

  Подставим \eqref{eq:complex-derivate-polar-2} в
  \eqref{eq:complex-derivate-polar-1} и получим

  \begin{equation*} \label{eq:complex-derivate-polar-3} \tag{3}
    f'(z) = \prh{\partder{u}{r} + i \partder{v}{r}} \cdot \frac{x}{r}
      + \prh{\partder{u}{\phi} + i \partder{v}{\phi}} \cdot \prh{-\frac{y}{r^2}}
  \end{equation*}
  
  Для второй скобки применим условия Коши-Римана в полярных координатах.

  \begin{equation*} \label{eq:complex-derivate-polar-4} \tag{4}
    \begin{aligned}
      f'(z)
      & = \prh{\partder{u}{r} + i \partder{v}{r}} \cdot \frac{x}{r}
      + \prh{-r \partder{v}{r} + i r \partder{u}{r}} \cdot \prh{-\frac{y}{r^2}}
    \\
      & = \prh{\partder{u}{r} + i \partder{v}{r}} \cdot \frac{x}{r}
      + \prh{\partder{v}{r} - i \partder{u}{r}} \cdot \frac{y}{r}
    \\
      & = \prh{\partder{u}{r} + i \partder{v}{r}} \cdot \frac{x}{r}
        - i \prh{\partder{u}{r} + i \partder{v}{r}} \cdot \frac{y}{r}
    \\
      & = \prh{\partder{u}{r} + i \partder{v}{r}}
        \cdot \prh{\frac{x}{r} - i \frac{y}{r}}
    \\
      & = \prh{\partder{u}{r} + i \partder{v}{r}} \cdot \frac{\bar{z}}{r}
    \\
      & = \prh{\partder{u}{r} + i \partder{v}{r}}
        \cdot \frac{\bar{z}}{r} \cdot \frac{z}{z}
    \\
      & = \prh{\partder{u}{r} + i \partder{v}{r}}
        \cdot \frac{r^2}{r z}
    \\
      & = \prh{\partder{u}{r} + i \partder{v}{r}}
        \cdot \frac{r}{z}
    \end{aligned}
  \end{equation*}
\end{proof}

\begin{remark}
  Аналогично доказательству выше можно показать, что

  \begin{equation*}
    f'(z) = \frac{1}{z} \prh{\partder{v}{\phi} - i \partder{u}{\phi}}
  \end{equation*}
\end{remark}

\begin{remark}
  Свойства дифференцируемых функций комплексного переменного полностью
  аналогичны свойствам дифференцируемых вещественных функций (дифференцируемость
  суммы, произведения, частного, композиции).
\end{remark}

\begin{example}
  Рассмотрим гладкую кривую в области \(D\): \(z = \phi(t) + i \psi(t) =
  \sigma(t)\).

  \begin{equation*}
    \omega'(t_0) = f'(z(t_0)) = f'(z_0) \sigma' (t_0)
  \end{equation*}

  Если потребуем \(f'(z_0) \neq 0\), то \(\omega'(t_0) \neq 0\).

  \begin{equation*}
    \begin{aligned}
      \Theta = \arg \sigma' (t_0)
      \qquad
      \widetilde{\Theta} = \arg \omega'(t_0)
    \\
      \under{\arg \omega'(t_0)}{\widetilde{\Theta}}
      = \arg f'(z_0) + \under{\arg \sigma'(t_0)}{\Theta}
    \\
      \widetilde{\Theta} - \Theta = \arg f'(z_0) = \alpha
    \end{aligned}
  \end{equation*}

  Таким образом \(\alpha\) это угол поворота. Он не зависит от вида кривой,
  значит функция \(\omega = f(z)\) сохраняет углы между кривыми.
\end{example}

\begin{example}
  Рассмотрим окружность \(\abs{z - z_0} = \rho\). Пусть \(\omega = f(z)\)~---
  аналитическая. Тогда

  \begin{equation*}
    \begin{aligned}
      \Delta z = f'(z_0) \Delta z + \smallo(\Delta z)
    \\
      \lim_{\Delta z \to 0} \abs{\frac{\Delta \omega}{\Delta z}}
      = \abs{f'(z_0)}
      \implies
      \abs{\Delta \omega}
      = \abs{\omega - \omega_0}
      = \abs{f'(z_0)} \under{\abs{z - z_0}}{\rho} + \smallo(\abs{\Delta z})
    \end{aligned}
  \end{equation*}

  Окружность \(\abs{z - z_0} = \rho\) переходит в окружность \(\abs{\omega -
  \omega_0}\) с постоянным растяжением \(\abs{f'(z_0)}\). Таким образом
  аналитическая функция сохраняет постоянство растяжения области \(D\) в область
  \(D'\).
\end{example}

\begin{definition}
  Отображение, сохраняющее углы  и постоянное растяжение, называется конформным.
\end{definition}

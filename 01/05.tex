\subsection{%
  Лекция \texttt{23.09.29}.%
}

\begin{theorem}[Интегрирование рядов]
  Пусть \(\sum u_n (x) = S(x)\)~--- мажорируем на \(\segment{a}{b}\). Тогда
  имеет смысл \(\int_a^x S(t) \dd t\) (\(\forall x \in \segment{a}{b}\)), а
  также

  \begin{equation*} \label{eq:ser-int-0} \tag{0}
    \int_a^x \prh[\Big]{\sum_{n = 1}^{\infty} u_n (t) \dd t}
    = \sum_{n = 1}^{\infty} \int_a^x u_n (t) \dd t
  \end{equation*}
\end{theorem}

\begin{proof}
  Если ряд мажорируем, то \(S(x)\) непрерывна и определен интеграл \(\int_a^x
  S(t) \dd t = A \in \RR\). Докажем равенство \eqref{eq:ser-int-0}. Пользуясь
  линейностью интеграла получаем

  \begin{equation*} \label{eq:ser-int-1} \tag{1}
    \begin{aligned}
      S(x) = \sum_{k = 1}^{n} u_k (x) + r_{n + 1} (x)
    \\
      \int_a^x S(t) \dd t
      = \int_a^x \prh[\Big]{u_1 (t) + \dotsc + u_n (t)} \dd t
        + \int_a^x r_{n + 1} (t) \dd t
      = \sum_{k = 1}^{n} \prh{\int_a^x u_k (t) \dd t}
        + \int_a^x r_{n + 1} (t) \dd t
    \end{aligned}
  \end{equation*}

  Далее поработаем с остатком ряда. Имеем \(\abs{r_{n + 1}} = \abs{u_{n + 1} (t)
  + u_{n + 2} (t) + \dotsc}\). Т.к. ряд мажорируем, то \(\forall i \colon
  \abs{u_i} \le a_i\), где \(\sum_{i = n + 1}^{\infty} a_i\) это остаток
  мажорирующего ряда.

  Т.к. мажорирующий ряд сходится, то его остаток \(\sum_{i = n + 1}^{\infty}
  a_i = \epsilon_i \Rarr{n \to \infty} 0\). Тогда \(\abs{r_{n + 1} (t)} <
  \epsilon_n \Rarr{n \to \infty} 0\). Получаем

  \begin{equation*} \label{eq:ser-int-2} \tag{2}
    \abs{\int_a^x r_{n + 1} (t) \dd t}
    \le \int_a^x \abs{r_{n + 1} (t)} \dd t
    < \int_a^x \epsilon_n \dd t
    = \epsilon_n (x - a)
    \Rarr{n \to \infty} 0
  \end{equation*}

  Таким образом \(\int_a^x r_{n + 1} (t) \dd t\) абсолютно сходится к нулю.
  Используя это и \eqref{eq:ser-int-1}, получаем

  \begin{equation*} \label{eq:ser-int-3} \tag{3}
    \begin{aligned}
      \lim_{n \to \infty} \int_a^x S(t) \dd t
      = \lim_{n \to \infty} \sum_{k = 1}^{n} \prh{\int_a^x u_k (t) \dd t}
        + \under{\lim_{n \to \infty} \int_a^x r_{n + 1} (t) \dd t}{= 0}
    \\
      \int_a^x S(t) \dd t
      = \sum_{k = 1}^{\infty} \prh{\int_a^x u_k (t) \dd t}
    \end{aligned}
  \end{equation*}

  Полученная формула разрешает почленно интегрировать мажорируемые ряды.
\end{proof}

\begin{theorem}
  \begin{equation*}
    \begin{rcases}
      \sum u_n (x) = S(x) \\
      \forall n \colon u_n \isdiff{\segment{a}{b}} \\
      \forall n \colon u'_n \iscont{\segment{a}{b}} \\
      \sum u'_n (x) \text{ мажорируем на } \segment{a}{b}
    \end{rcases}
    \implies
    \sum u'_n (x) = D(x) = S'(x)
  \end{equation*}
\end{theorem}

\begin{proof}
  \(D(x)\) непрерывна, т.к. ряд производных мажорируем. Тогда имеет смысл
  \(\int_a^x D(t) \dd t\) (\(\forall x \in \segment{a}{b}\)). Используя
  линейность интеграла получаем

  \begin{equation*}
    \begin{aligned}
      \int_a^x D(t) \dd t
      = \int_a^x u'_1 (t) \dd t + \int_a^x u'_2 (t) \dd t + \dotsc
      = \sum_{n = 1}^{\infty} u_n (t) \Big\rvert_a^x
      = \sum_{n = 1}^{\infty} \prh[\Big]{u_n (x) - u_n (a)}
      \eqby{\(\sum u_n(x) = S(x)\)} S(x) - \under{S(a)}{= const}
    \\
      D(x)
      = \prh{\int_a^x D(t) \dd t}'
      = S'(x) - \under{S'(a)}{= 0}
      = S'(x)
    \end{aligned}
  \end{equation*}
\end{proof}

\begin{remark}
  Не мажорируемые ряды формально дифференцируются и интегрируются почленно, но
  равенство сумм не выполняется (интеграл суммы не равен сумме интегралов).
\end{remark}

\begin{example}
  Рассмотрим следующий ряд

  \begin{equation*}
    \sum \frac{\sin n^4 x}{n^2}
    \qquad
    \abs{\frac{\sin n^4 x}{n^2}}
    \le \frac{1}{n^2}
    \implies
    \sum \frac{1}{n^2} \text{ мажорирующий}
  \end{equation*}

  Допустим, мы не проверили, что \(\sum u'_n (x)\) мажорируем и
  продифференцировали исходный ряд \quote{не глядя}.

  \begin{equation*}
    \begin{aligned}
      \sum u'_n (x)
      = \sum \frac{1}{n^2} \prh{\cos (n^4 x) \cdot n^4}
      = \sum n^2 \cos n^4 x
    \\
      u'_n = n^2 \under{\cos n^4 x}{\text{ограничена}}
      \Rarr{n \to \infty} \infty
    \end{aligned}
  \end{equation*}

  Мы видим, что не выполнятся необходимое условие сходимости.
\end{example}

\subheader{Степенные ряды}

\begin{definition}
  Ряд \(\sum_{n = 0}^{\infty} c_n x^n\) называется степенным рядом или рядом по
  степени \(x\).
\end{definition}

\begin{important}
  В степенных рядах обозначение \(\sum c_n x^n\) будет подразумевать, что нижняя
  граница равна нулю, а не единице, как это было ранее.
\end{important}

\begin{remark}
  Можно рассматривать степенные ряды со сдвигом в точку \(a\).

  \begin{equation*}
    \sum c_n (x - a)^n \eqby{\(x - a = t\)} \sum c_n t^n
  \end{equation*}
\end{remark}

\begin{theorem}[Абеля. Признак сходимости степенного ряда]
  Если \(\sum c_n x^n\) сходится в \(x_0 \neq 0\), тогда \(\forall x \colon
  \abs{x} < \abs{x_0}\) ряд сходится абсолютно и равномерно. Если \(\sum a_n
  x^n\) расходится в \(x_1\), тогда \(\forall x \colon \abs{x} > \abs{x_1}\) ряд
  расходится.
\end{theorem}

\begin{proof}
  Сначала докажем первую часть теоремы. Ряд \(\sum c_n x^n\) сходится в \(x_0
  \implies\) выполнено необходимое условие сходимости \(c_n x^n \Rarr{n \to
  \infty} 0\). Значит последовательность \(\seq{u_n}\) ограничена, т.е.
  \(\exists M > 0 \colon \abs{c_n x_0^n} \le M\). Рассмотрим ряд из модулей
  элементов исходного ряда

  \begin{equation*}
    \sum \abs{c_n x^n}
    = \abs{c_0} + \abs{c_1 x} + \abs{c_2 x^2} + \dotsc + \abs{c_n x^n} + \dotsc
    = \under{\abs{c_0}}{\le M}
      + \under{\abs{c_1 x_0}}{\le M} \cdot \abs{\frac{x}{x_0}}
      + \under{\abs{c_2 x_0^2}}{\le M} \cdot \abs{\frac{x^2}{x_0^2}}
      + \dotsc
    = \sum \abs{c_n x_0^n} \abs{\frac{x}{x_0}}^n
    \le \sum M \cdot \abs{\frac{x}{x_0}}^n
  \end{equation*}

  Т.к. \(\abs{x} < \abs{x_0}\), то этот ряд сходится \(\implies\) исходный ряд
  сходится абсолютно. Если \(\abs{x} < \abs{x_0}\), то \(\exists r > 0 \colon
  \abs{x} < r < \abs{x_0}\). Таким образом

  \begin{equation*}
    \abs{c_n x^n}
    \le M \cdot \abs{\frac{x}{x_0}}^n
    \le M \cdot \abs{\frac{r}{x_0}}^n
  \end{equation*}

  и \(M \abs{\frac{r}{x_0}}^n\)~--- члены мажорирующего ряда.

  Вторую часть теоремы докажем от противного. Пусть \(\exists \colon \abs{x} >
  \abs{x_1}\) и \(\sum c_n x^n\) сходится. Тогда согласно первой части теоремы в
  точке \(x_1\) ряд должен сходится. Противоречие.
\end{proof}

\begin{remark}
  Между интервалами сходимости и расходимости степенного ряда найдется точка
  \(\pm R\) называемая радиусом сходимости ряда. Интервал \(\interval{-R}{R}\)
  называется кругом сходимости.
\end{remark}

\begin{remark}
  В круге сходимости ряд мажорируем \(\implies\) интегрируем. Исследуем
  возможность дифференцирования. Нужно показать, что \(\sum u'_n (x) = \sum c_n
  \cdot n \cdot x^{n - 1}\) мажорируем. Рассмотрим интервал \(\interval{-r}{r}
  \in \interval{-R}{R}\). В этом интервале исходный ряд сходится, значит

  \begin{equation*}
    \begin{aligned}
      \forall \xi \in \interval{-R}{R} \setminus \interval{-r}{r}
      \implies
      \begin{cases}
        \abs{c_n \xi^n} \le M \in \RR^+ \\
        \frac{r}{\xi} < 1
      \end{cases}
    \\
      \forall x \in \interval{-r}{r} \colon
      \abs{u'_n (x)}
      = \abs{c_n \cdot n \cdot x^{n - 1}}
      \le \abs{c_n \cdot n \cdot r^{n - 1}}
      = \abs{\frac{c_n \xi^n}{\xi}}
        \cdot \abs{n \cdot \prh{\frac{r}{\xi}}^{n - 1}}
      \le \abs{\frac{M}{\xi}}
        \cdot \abs{n \cdot \prh{\frac{r}{\xi}}^{n - 1}}
    \end{aligned}
  \end{equation*}

  Если ряд из производных мажорируем, то ряд из вторых производных также
  мажорируем и т.д.
\end{remark}